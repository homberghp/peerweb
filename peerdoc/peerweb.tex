\documentclass[twocolumn,12pt,twoside]{article}
\usepackage{graphicx}
\usepackage{hyperref}
\usepackage{times}
\usepackage{amsmath}
\usepackage{rcs}
\RCSdef $RCSfile: styledef.tex,v $
\RCSdef $Author: hom $
\RCSdef $Revision: 1.1.1.1 $
\RCSdef $Date: 2008/10/24 18:00:48 $
\RCSdef $State: Exp $
\usepackage[a4paper,
% includemp,
      scale={0.92,0.93},
      includeheadfoot,
      bindingoffset=1.2cm]{geometry}
\usepackage{fancyhdr}
\title{Peerweb}
\setlength\columnseprule{1pt}
\setlength\columnsep{18pt}
\fancyfoot[LE,RO]{\includegraphics[height=1.5cm]{fon000_00c.pdf}}
\fancyfoot[LO,RE]{\vspace{-1.2cm}%
\tiny  File:\RCSRCSfile, \RCSState% version codes
  \\Author:\RCSAuthor%
  \\Version:\RCSRevision%
}
\setlength\unitlength{1mm}

\RCSdef $RCSfile: peerweb.tex,v $
\RCSdef $Author: hom $
\RCSdef $Revision: 1.1.1.1 $
\RCSdef $Date: 2008/10/24 18:00:48 $
\RCSdef $State: Exp $
\author{Pieter van den Hombergh\\FTHV}
\newcommand\peerweb{\textsf{Peerweb}{}}
\newcommand\Fontys{\textbf{Fontys}{}}
\begin{document}
%\sffamily
\pagestyle{fancy}
\lhead[\peerweb]{
\begin{picture}(0,0)(0,0)% punch holes
	\put(-7,-15){\circle{5}}%
	\put(-7,-95){\circle{5}}%
	\put(-13,-275){\rotatebox{90}{\tiny\RCSRCSfile, \RCSAuthor,\RCSRevision,\RCSDate}}%
	\put(-7,-175){\circle{5}}%
	\put(-7,-255){\circle{5}}%
	\end{picture}%
	\peerweb\ %
	}%
\rhead[User Guide]{User Guide}
\abstract{\peerweb\  is located at
\url{https://www.fontysvenlo.org/peerweb}.

Description and short userguide of \peerweb\ , 
to be used for peer assessment of the \Fontys Venlo
Campus. Functioning in the projects such as the Young Enterprises or Mini depends heavily on using the \Fontys
supplied email addresses. \textit{Other email addresses will not work}. Within
Mini \textit{all communication} will be done using these email
addresses. Everyone capable of using internet exploder or firefox can
use \url{webmail.fontys.nl}.}
\tableofcontents
\section{Introduction}
\peerweb\  is a web site of the \Fontys Technische Hogeschool Venlo that
started because our students objected to free riders in projects. 

Most of the project time at a modern University is spent without
any supervision by staff members, so
in order to get a fair evaluation of the individual effort, students
are requested to anonymously valuate each other on several aspects
with respect to the project. The tutors can use the input of the
fellow students as a 
suggestion to differentiate the grades given. The tutor will use the
grades given by the system as a suggestion. There are several methods,
to be used by the tutors, to ascertain the validity of the grades
given to the students. 

In most projects, students will have to share files, such as reports,
excel sheets, source code etc. This is where the \peerweb\  is also
helpful. You can use \peerweb\  to share files and even criticize those
files.  The uploaded files are protected from prying eyes by other
students during the time of the project. Afterwards, they can be read
by other students of the same module and year but of different project groups.

\peerweb\  also provides a simple mail service, which can be used to
communicate with your group fellows. 

Lastly, you can use \peerweb\  as a
timer to measure the amount of time you spend on your tasks, be it for
the project or for your personal tasks.

You can find \peerweb\  at \url{https://www.fontysvenlo.org/peerweb}.

\section{User guide}
To use \peerweb\  you will have to authenticate yourselves.

\subsection{Authentication}
The data of all students of Campus Venlo are already entered into
\peerweb\ .
The only thing you have to do is visit the \peerweb\  site and request
for a new authentication code. This can be done by entering your
student number and your birthday on the bottom part of the login
screen. Your password will be sent to your \Fontys email
address. It will be sent in a \texttt{.pdf} file, so make sure you can read or
print that. Each time you request a password you get a new one, but
always only sent to your \Fontys email address.

\subsection{Applying for participation to a Mini}
For the minis 2005-2006, you can apply for a
mini\footnote{mini=student company} by sending an email
to the email-address at fontysvenlo.org of the mini you will
participate in. This can be done 
during the first two weeks of September 2005\footnote{After that
period the email address can be used to send mail to other project members}
You should simply send an email \textit{from your \Fontys email
account} to the mini of your choice,
e.g. \verb#mini.12@fontysvenlo.org# if you want to participate in
mini team 12. \textbf{Note} that the word mini should be followed by a
period and the group number. The subject or content of the mail is
irrelevant, the sender's address and the receiver's address suffice.

\subsection{Login and logoff}
To be able to use the \peerweb\  you will have to login. Once you login
you will see the welcome page. There you will find some additional
text and some current data about your projects and the other members
of \peerweb\ . 

If you stop using a \peerweb\  session, please logoff by clicking the
cross \includegraphics[height=1em]{close_1.png} symbol on the top right hand of the pages. 

\subsection{The assessment proper}
Once the assessment has been set up by your tutor, you can fill the
assessment forms. Since there is little extra effort involved, you may
expect that you have to fill in an assessment typically four times per
year. Each such assessment is called a \textit{milestone}.

The form under tab \textbf{Assessment} is a simple table, which you have to fill in. 
Each row mentions a fellow group member and each column refers to a
criterion.
Please award a fair and just evaluation to your fellow students
on the criteria mentioned. The effects we observed in earlier years is
that the effort of all participants is optimized.

Once \textit{all} group members have filled in their form you can see
which \textit{average} grades you received from your fellows under the
tab \textbf{A-Results}. You will
not be able to see which grades the individual group members gave to
you.\footnote{Unless of course your group exists of only two members.}
You will receive an email as soon as the last member has filled in the
form. Under \textbf{A-results} you will find the following numbers: Under the column
\textbf{Grade} you find your individual average grade, calculated from the
grade your peers gave you on each criterion. Next, under \textbf{Group} you find the
average the group members gave all other group members. In the last
column \textbf{mult} you find the so called multiplier. If you did
well, $\text{mult} \geq 1$. This is used
as a suggestion for the tutor to distribute the group grade (if there
is a combined group product) among the individual members. Bottom left
hand you or the tutor can fill in the group grade, to get an 
impression of the individual grade you may receive. The default is a
7, but of course the tutor may give higher or lower group grades.

The tutor can also read these data, as well as a consolidation of the
grades for the whole group.

\subsection{Other functions}
At the time of this writing it is unclear if other functions should be
used within the mini. You are free to use them, but this is only
helpful if a whole mini team agrees on using them. You will have to find out how to use them by
yourself. There are a few bits of information on these functions on the
\peerweb\  website.
\section{Feedback}
Quality and usability of our service is a major concern of our organisation.
This means that suggestions to improve our service are always welcome.
\end{document}
